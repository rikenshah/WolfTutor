The application of WolfTutor is not actually specific to tutoring entirely.
It can reasonably be compared to any scheduling system.
For example, the appointment scheduling tool within NC State University's epack system functions similarly.
First you pick from an available list of appointment types. Then, you can opt to filter your search further by location, names of possible appointees, time of day, etc.
The tool will show available appointment time slots a user can book just like WolfTutor.
This type of scheduling mechanism is commonly used by other services as well e.g. medical facilities with multiple practitioners, personal trainers, life coaches, etc.

There are no mainstream applications relevant to automative tutor matching that give the users tutor recommendations. 
Most services only offer a place to find tutors manually, such as the 'Lessons' section on Craiglist. \cite{RefWorks:doc:5abd46a5e4b0770b05a4080c} 
There are services with more detail and structure than Cragislist like Verbling, an online application that contains profiles of Spanish tutors. 
These profiles contain more detailed and personal information about the user, including a photo. \cite{RefWorks:doc:5abd466ce4b0689719ee9277} 
Some services take other approaches. 
The popular tutoring service, Chegg, does not facilitate scheduling or tutoring sessions, but instead lets multiple tutors cater to a student's question. \cite{RefWorks:doc:5abd45f7e4b0770b05a407c4}



%%% Local Variables:
%%% mode: latex
%%% TeX-master: "../../../main"
%%% End:
