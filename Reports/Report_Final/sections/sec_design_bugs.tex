The existing code was of a reasonable quality at the time that it was handed off
to group-p.  That said, there were a few issues that needed to be addressed
before work could proceed.

The first and most glaring is in the documentation.  The documentation provided
to group-p was actually quite good.  The team was able to get an application up
and running with the basic specification within a few hours, but one step was
left out that prevented any registration as a tutor or searching for tutors: a
database entry had to be created for at least one subject manually.  Once
the team reached out to the original developers of WolfTutor, the problem was
rectified within 24 hours and the appropriate documentation was updated.

% TODO: is this right?  I can't recall.
The next bug that had to be fixed was in the review process.  For whatever
reason, the exisitng review functionality was nonfunctional when tested on
group-p's development environments, and some time had to be spent getting that
functionality working for testing.

Lastly, there was a very minor issue that would cause a timeout message to
appear every time the bot was restarted and a new student would start
interacting with it.  This bug did not break the software, but the team did
spend a little time to root it out just the same.
%%% Local Variables:
%%% mode: latex
%%% TeX-master: "../main"
%%% End:
