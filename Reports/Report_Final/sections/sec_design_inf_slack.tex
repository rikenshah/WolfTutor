\paragraph{Slack}
Slack is primarily a business tool intended to help facilitate
communication and coordination between individuals and groups at organizations.
In practice, however, Slack has become a familiar name with many unintended audiences such
as online communities and even university courses such as this one because of it's 
availability and ease of use. Attracting users to a platform they're already familiary with
is generally easier and less costly than attracting attention to a brand new website.

Another reason for using the Slack platform is that it already supports the languages and keyboard layouts for English, 
Japanese, German, Spanish, and French. Their website also claims to be 
including more regions in the future. Because of this, using a platform like Slack 
is beneficial to smaller developers as they can spend less time building a unique website 
and pushing it out internationally and more time on the functionality of the product. 

\paragraph{Slack Bot}
As mentioned previously, the target for this chat bot is to run in the Slack
platform. The Slack bot handles the interactions between the user and the NodeJS applications. 
All responses given by the user would be posted to the application by the bot. 
Similarily, all prompts sent to the user from application would be posted to Slack by the bot.

While it is true that this bot has been deployed to Slack specifically, it was
developed using a popular API called Botkit, which actually has APIs for
multiple chat services such as Facebook Messenger or Discord as well as Slack.
Because of this, it is possible to port WolfTutor to these other platforms
without needing to entirely start over.

%%% Local Variables:
%%% mode: latex
%%% TeX-master: "../main"
%%% End:
