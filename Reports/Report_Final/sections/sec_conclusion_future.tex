\subsubsection{Further Evaluation}
\label{sec:further-evaluation}
The team has made a best effort to evaluate the algorithm presented in this
paper both using a manually-labeled data set and using in-person interviews with
potential users. That said, true evaluation of educational software must be done
in a classroom setting.  

% TODO: describe ideal study
A good way to evaluate the usefullness of this system would be to truly put it
in the hands of students over a period of time. Over the course of several
semesters, a different set of sections of one course should be evaluated.  For
this reason, it may make sense to use a popular core course to a given major,
such as Intro to Computer Science.  These sections could be grouped into 3
groups, one with no intervention, one where students are pushed to use the
university-sponsored tutoring facilities that exist today, and one where
students are given access to the same facilities but are also given access to
WolfTutor.  Interviews would need to be conducted after student tutoring
sessions throughout the semester asking the students for feedback about the
quality of their tutoring session specifically and features of the application
more broadly.  At the end of the semester(s), the grades of the different groups
could be compared to historical data for the course and some generalizations
could be made about the usefulness of the software.

It is important to note, however, that an improvement in grades may not be the
right measure of success for this system. While better grades would certainly be
ideal, simply higher rates of engagement between students and tutors would be
one reasonable measure of success. Also, reducing time spent scheduling tutoring
sessions between students in the traditional tutoring section and WolfTutor
students would also be a reasonable win for students.  

\subsubsection{Future Enhancement}
% TODO: describe the process of implementing a new algorithm in this setting
\label{sec:future-enhancement}

%%% Local Variables:
%%% mode: latex
%%% TeX-master: "../main"
%%% End:
