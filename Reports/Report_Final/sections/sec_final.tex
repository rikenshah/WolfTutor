The final thought to conisder is whether or not group-p's contribution to
WolfTutor is truly an improvement over the existing system. In most fields, the
combination of the algorithmic evaluation provided in section \ref{sec:evaluation}
and the use validation provided in \ref{sec:evaluation} would be sufficent to
argue that the system is an improvement, but the education field is one that has
higher stakes than most.  

Educational data miners have spent considerable ink discussing the various
pitfalls in identifying high and low performers algorithmically, and the problem
of tutor recommendation is something that could potentially also be problematic.
If a student is connected with a poor tutor or one that handles their position
of power poorly, the potential impact could be catastrophic.  This may help
explain why many universities (such as Marshall University) require students to
take the courses they tutor for and bar anyone with lower than a B in that
subject from tutoring on top of requiring an in-person interview.

So does WolfTutor do enough to guarantee that a tutor is good or that they will
work well with the students?  The evaluation that has been done so far points
that it will make a good-effort to make recommendations that will ``do no harm''
but it would probably still make school administrators uneasy.  When combined
with a university vetting process, though, WolfTutor, especially after the
application of group-p's recommendations could serve universities and other
levels of schooling well as a platform for connecting students to peer tutors.  

%%% Local Variables:
%%% mode: latex
%%% TeX-master: "../main"
%%% End:
