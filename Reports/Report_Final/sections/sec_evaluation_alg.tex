As we implement the enhancement of wolfTutor during each sprint, 
a variety of test were conducted to verify the functionality and performance. 
For example, 1) unit tests were generated after the implementation of some 
core part, like prioritize method, so that to make sure the functionality of
 new feature. 2) We also conducted a full user case manually tests after 
 each sprint, to verify the functionality of the whole application. This
  means we fire up the server, mock the process by enrolling to the system, 
  becoming a tutor, switching to another user account, finding a tutor, 
  reviewing a prvious tutor session, and also loading the history of prvious tutor sessions.

In order to verify tutor matching algorithm quantitively, we need a database which contains  
%%% Local Variables:
%%% mode: latex
%%% TeX-master: "../main"
%%% End:
